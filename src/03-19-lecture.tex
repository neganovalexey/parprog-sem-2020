\documentclass[aspectratio=169, pdf, 8pt, unicode]{beamer}
\usepackage[american,russian]{babel}
\usepackage[default]{sourcesanspro}
\usepackage{float}
\usepackage{graphicx}
\usepackage{pgfplotstable}
\usepackage{caption}
\usepackage{amsmath}
\usepackage{amssymb}
\usepackage{graphicx}
\usepackage{tikz}
\usetikzlibrary{automata, positioning, arrows, chains, fit, shapes}
\usepackage{tikzscale}
\usepackage{setspace}
\usepackage{fancyvrb}

\DeclareCaptionLabelFormat{gostfigure}{Рисунок #2}
\captionsetup[table]{labelsep=endash,justification=justified,singlelinecheck=false,font=normalsize,skip=0pt} 
\captionsetup[figure]{labelformat=gostfigure,labelsep=endash,justification=centering,singlelinecheck=false,font=normalsize} 
\pgfplotsset{compat=1.9}

\mode<presentation> {
\usetheme{Madrid}
}

\setbeamerfont{institute}{size=\normalsize}
\setbeamertemplate{itemize/enumerate body begin}{\large}
\setbeamertemplate{itemize/enumerate subbody begin}{\tiny}

\title[Теория и практика многопоточного программирования]{Теория и практика многопоточного программирования\\ \vspace{0.5cm}Лекция 6}

\author{Неганов Алексей}

\institute[МФТИ]{
    Московский физико-технический институт (национальный исследовательский университет)\\
    Кафедра теоретической и прикладной информатики\\
}

\date{Москва 2020}

\setbeamertemplate{caption}[numbered]

\begin{document}

\begin{frame}
\titlepage
\end{frame}

\begin{frame}[fragile]
\frametitle{Формализм конечного автомата}
\begin{block}{Определение}
    \textbf{Недетерминированный конечный автомат (НКА)} --- это пятёрка
    \begin{equation}
        M = (Q, \Sigma, \delta, q_0, F),
    \end{equation}
    где $Q$ --- конечное множество состояний,\\
    $\Sigma$ --- конечное множество допустимых входных символов (входной алфавит),\\
    $\varepsilon$ --- пустая цепочка символов,\\
    $\delta \colon \: Q \times (\Sigma \cup \{\varepsilon\}) \rightarrow P(Q)$ --- функция переходов, отображающая пару
        <<состояние -- символ>> в некое подмножество $Q$,\\
    $q_0 \in Q$ --- начальное состояние,\\
    $F \subseteq Q$ --- множество заключительных (допускающих) состояний.
\end{block}
\end{frame}

\begin{frame}[fragile]
\frametitle{Формализм конечного автомата}
\begin{figure}[H]
    \centering
    \begin{tikzpicture}
        \node[state, initial] (1) {$1$};
        \node[state, below left of=1, xshift=-1cm, yshift=-1cm] (2) {$2$};
        \node[state, accepting, right of=2, xshift=1cm] (3) {$3$};
        \draw [->] (1) edge[above] node{$b$} (2)
                   (1) edge[below, bend right, left=0.3] node{$\varepsilon$} (3)
                   (2) edge[loop left] node{$a$} (2)
                   (2) edge[below] node{$a, b$} (3)
                   (3) edge[above, bend right, right=0.3] node{$a$} (1);
    \end{tikzpicture}

    \vspace{1cm}

    \begin{tikzpicture}
        \tikzstyle{every path}=[very thick]
        \edef\sizetape{0.7cm}
        \tikzstyle{tmtape}=[draw,minimum size=\sizetape]
        \tikzstyle{tmhead}=[arrow box,draw,minimum size=.5cm,arrow box arrows={east:.25cm, west:0.25cm}]

        %% Draw TM tape
        \begin{scope}[start chain=1 going right,node distance=-0.15mm]
            \node [on chain=1,tmtape,draw=none] {$\ldots$};
            \node [on chain=1,tmtape] {};
            \node [on chain=1,tmtape] (input) {b};
            \node [on chain=1,tmtape] {b};
            \node [on chain=1,tmtape] {a};
            \node [on chain=1,tmtape] {a};
            \node [on chain=1,tmtape] {a};
            \node [on chain=1,tmtape] {a};
            \node [on chain=1,tmtape] {};
            \node [on chain=1,tmtape,draw=none] {$\ldots$};
        \end{scope}

        %% Draw TM head below (input) tape cell
        \node [tmhead,yshift=-.3cm] at (input.south) (head) {$q$};
    \end{tikzpicture}
    \caption{Диаграмма состояний недетерминированного конечного автомата}
\end{figure}
\end{frame}

\begin{frame}[fragile]
\frametitle{Модель потока}
\frametitle{Формализм конечного автомата: модель потока}
\begin{itemize}
%\setlength\itemsep{2em}
    \item $Q$ --- состояние памяти и регистров процессора,
    \item $q_0$ --- точка начала исполнения потока \textit{(entry point)},
    \item Переход между состояниями $(q_i, q_j)$ --- событие \textit{(event)},
        причём каждому событию ставится в соответствие точка на временной шкале,
    \item Все события по определению одномоментны (квантование времени).
    \item Никакие два события не являются одновременными.
    \item События $a$ и $b$ называются \textbf{упорядоченными,} если $a$ предшествует $b$: $a \to b$.
    \item Произвольные два события $a$ и $b$ могут быть неупорядочены: $a \not\to b$ и $b \not\to a$.
    \item Отношение порядка транзитивно: если $a \to b$ и $b \to c$, то $a \to c$.
    \item \textbf{Интервал} $I(a,b)$ --- это пара событий $(a, b)$ такая, что $a \to b$.\\
    \item Интервал $I(a,b)$ \textbf{предшествует} интервалу $I(b, c)$ ($I(a,b) \to I(b,c)$), если $b \to c$.
    \item Интервалы, не связанные отношением <<$\to$>>, называются \textbf{соисполняемыми} \textit{(concurrent)}.
\end{itemize}

Пример простого доказательства корректности для lock-free hash table\\
\texttt{https://docs.huihoo.com/javaone/2007/java-se/TS-2862.pdf}

Ключевые слова для желающих углубиться: discrete event simulation, communicating finite-state machines
\end{frame}

\begin{frame}[fragile]
\frametitle{Формализм конечного автомата: модель потока}
\begin{figure}[H]
\centering
\begin{minipage}{0.8\textwidth}
\small
\begin{verbatim}
#include <thread>
#include <atomic>
#include <stdio.h>
#include <vector>

std::atomic<int> a, b; // state: variables

void f1() {
    a = 0xF; // event: variable assignment
    while (a != b); // event: wait for condition
    printf("OK\n");
}

void f2(int c) {
    b |= c; // event: variable assignment
}

int main() {
    b.store(0);
    std::vector<std::thread> v(5);
    v[4] = std::thread(f1);
    for (int i = 0; i < 4; i++)
        v[i] = std::thread(f2, 1 << i);
    for (auto &t: v)
        t.join();
    printf("That's all\n");
}
\end{verbatim}
\end{minipage}
\end{figure}
\end{frame}

\begin{frame}
\frametitle{Взаимоисключение и критические секции}
\begin{block}{Определение}
   Назовём интервал $CS_i$ \textbf{критической секцией}, если для детерминированности совместного исполнения потоков необходимо,
   чтобы для любых потоков $A$, $B$ их критические секции были упорядочены: $CS_A \to CS_B$ или $CS_B \to CS_A$.
   Выполнение этого условия для системы потоков называется свойством взаимоисключения \textit{(mutual exclusion)}.
\end{block}
\begin{block}{Критерий Бернстайна}
	Пусть $R(P_i)$ множество переменных, значение которых поток $P_i$ использует в операциях чтения, $W(P_i)$ ---
	множество переменных, использующихся в операциях записи. Тогда совокупное исполнение потоков $P_1$ и $P_2$ детерминировано, если
	\begin{equation}
	\left\{
	\begin{aligned}
		W(P_1) \cap W(P_2) = \varnothing \\
		R(P1)  \cap W(P_2) = \varnothing \\
		W(P_1) \cap R(P_2) = \varnothing \\
	\end{aligned}
	\right.
   \end{equation}
\end{block}
\end{frame}

\begin{frame}
\frametitle{Свойства алгоритмов}
\begin{enumerate}
\item \textbf{Отсутствие зависаний} \textit{(starvation freedom / lockout freedom)}\\
    Каждый поток рано или поздно попадает в критическую секцию, т.~е. каждое обращение к методам рано или поздно завершается.
\item \textbf{Неблокируемость} \textit{(non-blocking)}\\
    Блокировка (задержка) одного потока не задерживает другие потоки.
\item \textbf{Свобода от взаимной блокировки} \textit{(freedom from deadlock)}\\
    Хотя бы один из конкурирующих потоков должен рано или поздно попасть в нужную ему критическую секцию. Если некоторый поток не может войти в критическую секцию, то другие потоки должны завершить бесконечное количество критических секций.
\item \textbf{Честность} \textit{(fairness)}\\
    Если $D_A$ и $D_B$ --- попытки входа (взятия блокировки) в критические секции $CS_A$ и $CS_B$ для потоков $A$ и $B$ и $D_A \to D_B$, то
    $CS_A \to CS_B$.
\item \textbf{Свобода от ожидания} \textit{(wait-free)}\\
    Выполнение метода заканчивается за конечное количество шагов без каких-либо процедур ожидания.
\item \textbf{Свобода от блокировок} \textit{(lock-free)}\\
    Неограниченно частый вызов метода завершится за конечное число шагов.
\end{enumerate}
\end{frame}

\begin{frame}[fragile]
\frametitle{Проблемы блокировок}
\begin{itemize}
\setlength\itemsep{2em}
\item Крупнозернистая синхронизация плохо масштабируется
\item Мелкозернистая тяжела в написании
\item Тупики
\item Инверсия приоритета
\item Конвоирование
\end{itemize}
\end{frame}

\begin{frame}[fragile]
\frametitle{wait-free}
\begin{itemize}
\item Гарантия безусловного прогресса для каждого потока в отдельности
\item Гарантия безусловного прогресса системы в целом
\item Обычно используются блокирующие память операции
\item Реализации часто требуют $O(N)$ памяти, где $N$ — число потоков
\end{itemize}
\begin{figure}[H]
\centering
\begin{minipage}{0.8\textwidth}
\begin{verbatim}
someClass::someClass() {
    ...
    __sync_fetch_and_add(referenceCount, 1);
}

someClass::~someClass() {
    ...
    if (__sync_sub_and_fetch(referenceCount, 1) == 0) {
        // delete unused shared object
    }
}
\end{verbatim}
\end{minipage}
\end{figure}
\end{frame}

\begin{frame}[fragile]
\frametitle{lock-free}
\begin{itemize}
\item Уже нет гарантии безусловного прогресса для каждого потока в отдельности.
\item Гарантия безусловного прогресса системы в целом, то есть хотя бы один поток продвигается вперёд независимо от внешних факторов
\item Остальные потоки могут в это время активно ожидать, то есть, непродуктивно использовать процессорное время.
\item Обычно используются CAS-примитивы.
\item Может наблюдаться инверсия приоритетов и конвоирование.
\item Контейнеры, как правило, быстрее wait-free.
\end{itemize}
\begin{figure}[H]
\centering
\begin{minipage}{0.8\textwidth}
\begin{verbatim}
void push(struct node *n) {
      do {
            struct node *t = top;
            n->next = t;
      while (!CAS(&top, t, n));
}
\end{verbatim}
\end{minipage}
\end{figure}
\end{frame}

\begin{frame}[fragile]
\frametitle{obstruction-free}
\begin{itemize}
    \item Условный прогресс: поток продвигается вперёд только тогда, когда нет конкуренции со стороны других потоков.
    \item Система в целом при большой взаимной конкуренции потоков не продвигается вперёд.
    \item Возможна живая блокировка (live-lock).
    \item Алгоритмы могут быть быстрее других.
    \item Некоторые алгоритмы можно реализовать лишь так (double-linked list).
\end{itemize}
\end{frame}

\begin{frame}[fragile]
\frametitle{pure-free}
\begin{itemize}
    \item Свободный метод для изолированного потока обеспечивает завершение за конечное число шагов.
    \item Не гарантирует совместный прогресс.
    \item При наличии других потоков возможна взаимная блокировка.
    \item Все полны оптимизма, работают на себя и надеются на лучшее.
\end{itemize}
\end{frame}

\end{document}
