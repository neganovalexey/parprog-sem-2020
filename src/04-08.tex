\documentclass[aspectratio=169, pdf, 8pt, unicode]{beamer}
\usepackage[american,russian]{babel}
\usepackage[default]{sourcesanspro}
\usepackage{float}
\usepackage{graphicx}
\usepackage{pgfplotstable}
\usepackage{caption}
\usepackage{amsmath}
\usepackage{amssymb}
\usepackage{setspace}
\usepackage{fancyvrb}
\usepackage[outputdir=aux]{minted}

\DeclareCaptionLabelFormat{gostfigure}{Рисунок #2}
\captionsetup[table]{labelsep=endash,justification=justified,singlelinecheck=false,font=normalsize,skip=0pt} 
\captionsetup[figure]{labelformat=gostfigure,labelsep=endash,justification=centering,singlelinecheck=false,font=normalsize} 
\pgfplotsset{compat=1.9}

\mode<presentation> {
\usetheme{Madrid}
}

\setbeamerfont{institute}{size=\normalsize}
\setbeamertemplate{itemize/enumerate body begin}{\large}
\setbeamertemplate{itemize/enumerate subbody begin}{\tiny}

\title[Теория и практика многопоточного программирования]{Теория и практика многопоточного программирования\\ \vspace{0.5cm}Семинар 9}

\author{Неганов Алексей}

\institute[МФТИ]{
    Московский физико-технический институт (национальный исследовательский университет)\\
    Кафедра теоретической и прикладной информатики\\
}

\date{Москва 2020}

\setbeamertemplate{caption}[numbered]

\begin{document}

\begin{frame}
\titlepage
\end{frame}

\begin{frame}[fragile]
\frametitle{Monitor}
\begin{figure}[H]
\centering
\begin{minipage}{0.8\textwidth}
\begin{minted}{Java}
acquire(m); // Acquire this monitor's lock.
while (!p) { // While the condition/predicate/assertion that we are waiting for is not true...
	wait(m, cv); // Wait on this monitor's lock and condition variable.
}
// ... Critical section of code goes here ...
signal(cv2); -- OR -- notifyAll(cv2); // cv2 might be the same as cv or different.
release(m); // Release this monitor's lock.
\end{minted}
\end{minipage}
\end{figure}
\end{frame}

\begin{frame}[fragile]
\frametitle{Seqlock}
\begin{figure}[H]
\begin{minipage}{0.4\textwidth}
\begin{minted}{C}
void thread_read(void) {
    do {
        while((c=lock->cnt)&1);
        /* read */
    } while(lock->cnt!=c);
}
\end{minted}
\end{minipage}
\begin{minipage}{0.4\textwidth}
\begin{minted}{C}
void thread_write(void) {
    do {
        while((c=lock->cnt)&1);
    } while(!CAS(&lock->cnt,c,c+1);
    /* write */
    atomic_fetch_add(&lock->cnt, 1);
}
\end{minted}
\end{minipage}
\end{figure}
\begin{itemize}
\item нет указателей
\item частые чтения
\item запись редка, но обязана быть быстрой
\end{itemize}
\end{frame}

\begin{frame}[fragile]
\frametitle{Krefs}
\begin{figure}[H]
\begin{minipage}{0.4\textwidth}
\begin{minted}{C}
struct foo {
    struct kref kref;
}

struct foo *foo;
foo = kmalloc(sizeof(*foo), GFP_KERNEL);
kref_init(&foo->kref, foo_release);
kref_get(&foo->kref);
kref_put(&foo->kref);

struct kref *kref_get(struct kref *kref) {
    atomic_inc(&kref->refcount);
    return kref;
}

void foo_release(struct kref *kref) {
    struct foo *foo;
    foo = container_of(foo, struct foo, kref);
    kfree(foo);
}
\end{minted}
\end{minipage}
\end{figure}
\end{frame}

\begin{frame}[fragile]
\frametitle{Fastflow}
\begin{figure}[H]
\begin{minipage}{0.4\textwidth}
\begin{minted}{C++}
Bool push(Element& item) {
    int nextTail = increment(tail);
    if (nextTail != head) {
        array[tail] = item;
        tail = nextTail;
        return true;
    }
    return false;
}

Bool pop(Element& item) {
    if (tail == head)
        return false;
    item = array[head];
    head = increment(head);
    return true;
}
\end{minted}
\end{minipage}
\end{figure}
\end{frame}

\end{document}
