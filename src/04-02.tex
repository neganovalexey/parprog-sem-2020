\documentclass[aspectratio=169, pdf, 8pt, unicode]{beamer}
\usepackage[american,russian]{babel}
\usepackage[default]{sourcesanspro}
\usepackage{float}
\usepackage{graphicx}
\usepackage{pgfplotstable}
\usepackage{caption}
\usepackage{amsmath}
\usepackage{amssymb}
\usepackage{setspace}
\usepackage{fancyvrb}
\usepackage[outputdir=aux]{minted}

\DeclareCaptionLabelFormat{gostfigure}{Рисунок #2}
\captionsetup[table]{labelsep=endash,justification=justified,singlelinecheck=false,font=normalsize,skip=0pt} 
\captionsetup[figure]{labelformat=gostfigure,labelsep=endash,justification=centering,singlelinecheck=false,font=normalsize} 
\pgfplotsset{compat=1.9}

\mode<presentation> {
\usetheme{Madrid}
}

\setbeamerfont{institute}{size=\normalsize}
\setbeamertemplate{itemize/enumerate body begin}{\large}
\setbeamertemplate{itemize/enumerate subbody begin}{\tiny}

\title[Теория и практика многопоточного программирования]{Теория и практика многопоточного программирования\\ \vspace{0.5cm}Семинар 8}

\author{Неганов Алексей}

\institute[МФТИ]{
    Московский физико-технический институт (национальный исследовательский университет)\\
    Кафедра теоретической и прикладной информатики\\
}

\date{Москва 2020}

\setbeamertemplate{caption}[numbered]

\begin{document}

\begin{frame}
\titlepage
\end{frame}

\begin{frame}[fragile]
\frametitle{Задача производителя-потребителя}
\begin{figure}[H]
\begin{minipage}{0.4\textwidth}
\small
\begin{minted}{Go}
package main
import "fmt"

var done = make(chan bool)
var msgs = make(chan int)

func produce () {
    for i := 0; i < 10; i++ {
        msgs <- i
    }
    done <- true
}

func consume () {
    for {
      msg := <-msgs
      fmt.Println(msg)
   }
}

func main () {
   go produce()
   go consume()
   <- done
}
\end{minted}
\end{minipage}
\end{figure}
\end{frame}


\begin{frame}[fragile]
\frametitle{Задача производителя-потребителя}
\begin{figure}[H]
\begin{minipage}{0.4\textwidth}
%\small
\begin{minted}{C}
int itemCount = 0;

void producer(void) {
    while (1) {
        item = produceItem();
        if (itemCount == BUFFER_SIZE)
            sleep();

        putItemIntoBuffer(item);
        itemCount = itemCount + 1;

        if (itemCount == 1)
            wakeup(consumer);
    }
}
\end{minted}
\end{minipage}
\begin{minipage}{0.4\textwidth}
%\small
\begin{minted}{C}
void consumer(void) {
    while (1) {
        if (itemCount == 0)
            sleep();

        item = removeItemFromBuffer();
        itemCount = itemCount - 1;

        if (itemCount == BUFFER_SIZE - 1)
            wakeup(producer);

        consumeItem(item);
    }
}
\end{minted}
\end{minipage}
\end{figure}
Почему такой код не является корректным?
\end{frame}

\begin{frame}[fragile]
\frametitle{Семафоры}
\begin{figure}[H]
\centering
\begin{minipage}{0.4\textwidth}
\begin{verbatim}
# ALGOL up
# POSIX sem_post
function V(semaphore S, integer I):
    [S ← S + I]

# ALGOL down
# POSIX sem_wait
function P(semaphore S, integer I):
    repeat:
        [if S ≥ I:
        S ← S − I
        break]
\end{verbatim}
\end{minipage}
\end{figure}
Операции $P$ и $V$ имеет право делать \textbf{любой} поток, в отличие от \texttt{mutex}.
\end{frame}

\begin{frame}[fragile]
\frametitle{Задача производителя-потребителя: семафоры}
\begin{figure}[H]
\begin{minipage}{0.4\textwidth}
%\small
\begin{minted}{C}
sem_t fill, empty;
sem_init(&fill, 0, 0);
sem_init(&empty, 0, BUFFER_SIZE);

void producer(void) {
    while (1) {
        item = produceItem();
        sem_wait(&empty);
        putItemIntoBuffer(item);
        sem_post(&fill);
    }
}
\end{minted}
\end{minipage}
\begin{minipage}{0.4\textwidth}
%\small
\begin{minted}{C}
procedure consumer() {
    while (1) {
        sem_wait(&fill);
        item = removeItemFromBuffer();
        sem_post(&empty);
        consumeItem(item);
    }
}
\end{minted}
\end{minipage}
\end{figure}
\end{frame}

\begin{frame}[fragile]
\frametitle{Задача производителя-потребителя: семафоры}
\begin{figure}[H]
\begin{minipage}{0.4\textwidth}
%\small
\begin{minted}{C}
pthread_mutex_t lock;
sem_t fill, empty;
sem_init(&fill, 0, 0);
sem_init(&empty, 0, BUFFER_SIZE);

void producer(void) {
    while (1) {
        item = produceItem();
        sem_wait(&empty);
        pthread_mutex_lock(&lock);
        putItemIntoBuffer(item);
        pthread_mutex_unlock(&lock);
        sem_post(&fill);
    }
}
\end{minted}
\end{minipage}
\begin{minipage}{0.4\textwidth}
%\small
\begin{minted}{C}
procedure consumer() {
    while (1) {
        sem_wait(&fill);
        pthread_mutex_lock(&lock);
        item = removeItemFromBuffer();
        pthread_mutex_unlock(&lock);
        sem_post(&empty);
        consumeItem(item);
    }
}
\end{minted}
\end{minipage}
\end{figure}
\end{frame}

\begin{frame}[fragile]
\frametitle{Задача читателя-писателя: приоритет читателя}
\begin{figure}[H]
\begin{minipage}{0.4\textwidth}
\begin{minted}{C}
semaphore resource=1;
semaphore rentry=1;

int rcount=0;

writer() {
    P(resource);
    // write
    V(resource)
}
\end{minted}
\end{minipage}
\begin{minipage}{0.4\textwidth}
\begin{minted}{C}
reader() {
    P(rentry);
    readcount++;
    if (++rcount == 1)
        P(resource);
    v(rentry)

    // read

    P(rentry)
    if (--rcount == 0)
        V(resource);
    V(rentry);
}
\end{minted}
\end{minipage}
\end{figure}
См. \texttt{pthread\_rwlock\_t}
\end{frame}

\begin{frame}[fragile]
\frametitle{Задача читателя-писателя: приоритет писателя}
\begin{figure}[H]
\begin{minipage}{0.4\textwidth}
\begin{minted}{C}
int rcount = 0, wcount = 0;
semaphore rentry = 1, wentry = 1;
semaphore readTry = 1, resource = 1;

writer() {
    P(wentry);
    if (++wcount == 1)
        P(readTry);
    V(wentry);

    P(resource);
    // write
    V(resource);

    P(wentry);
    if (--writecount == 0)
        V(readTry);
    V(wentry);
}
\end{minted}
\end{minipage}
\begin{minipage}{0.4\textwidth}
\begin{minted}{C}
reader() {
    P(readTry);
    P(rentry);
    if (++readcount == 1)
        P(resource);
    V(rentry);
    V(readTry);

    // read

    P(rentry);
    if (--readcount == 0)
        V(resource);
    V(rentry)
}
\end{minted}
\end{minipage}
\end{figure}
\end{frame}

\begin{frame}[fragile]
\frametitle{Задача читателя-писателя: честное решение}
\begin{figure}[H]
\begin{minipage}{0.4\textwidth}
\begin{minted}{C}
int rcount = 0;
semaphore rentry = 1, resource = 1;
semaphore queue = 1;

writer() {
    P(queue);
    P(resource);
    V(queue);

    // write

    V(resource);
}
\end{minted}
\end{minipage}
\begin{minipage}{0.4\textwidth}
\begin{minted}{C}
reader() {
    P(queue);
    P(rentry);
    if (++readcount == 1)
        P(resource);
    V(rentry);
    V(queue);

    // read

    P(rentry);
    if (--readcount == 0)
        V(resource);
    V(rentry);
}
\end{minted}
\end{minipage}
\end{figure}
\end{frame}

\begin{frame}[fragile]
\frametitle{Задачи}
\begin{enumerate}
\item Реализовать RW lock с помощью семафоров POSIX
\item Реализовать RW lock с помощью атомарных примитивов
\item Реализовать передачу содержимого файла из одного процесса в другой через разделяемый буфер, используя семафоры. Требование: смерть любого процесса 
должна корректно обработаться.
\end{enumerate}
\end{frame}

\end{document}
